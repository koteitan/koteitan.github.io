\documentclass[a4j, 12pt]{jsarticle}
\usepackage{amsmath,amssymb,bm,ascmac}
\usepackage[dvipdfmx]{hyperref}
\usepackage{pxjahyper}
\hypersetup{% hyperref
setpagesize=false,
 bookmarksnumbered=true,%
 bookmarksopen=true,%
 colorlinks=true,%
}
\usepackage[top=10truemm,bottom=20truemm,left=8truemm,right=8truemm]{geometry}
\usepackage{comment}
\renewcommand{\liminf}{\lim_{d\rightarrow\infty}}
\newcommand{\nat}{\mathbb{N}}
\renewcommand{\P}{P_{k,d}(x)}
\newcommand{\Q}{Q_{k,d}(x)}
\newcommand{\R}{R_{k,d}(x)}
\renewcommand{\d}{\partial}
\author{koteitan}
\title{ネストすごろく2 ~一般化ネストすごろく~}

\begin{document}
\maketitle

\section{問題}
\begin{shadebox}
  [問2] サイコロの目の数 $d$ を増やしていくと、「確率 1 でクリア出来るネストすごろく\cite{nestsugoroku}の最大の nest マスの数」の極限はどうなるでしょうか?($d$ の関数で表してみて下さい)\cite{nestsugoroku2}
\end{shadebox} 
~

\section{解法}
\subsection{クリア確率 $\P$}
サイコロの目の数が $d$ であり,ゴールまで nest マスの残りが $k$ 個であり、nest マスに停まったときのミニゲームをクリアできる確率がx であるネストすごろくをクリアできる確率を $\P$ としてみましょう.つまり,下記のような状況ですね.
\begin{equation*}
  \textrm{[今いる場所]} \underbrace{\textrm{[nest] [nest] ... [nest] [nest]}}_{k} \textrm{[GOAL]}
\end{equation*}
さらに,ゴールマスを行き過ぎてもゴール扱いされるルールのことを考えると,こう表すとより分かりやすいです.
\begin{equation*}
  \textrm{[今いる場所]} \overbrace{\underbrace{\textrm{[nest] [nest] ... [nest] [nest]}}_{k} \underbrace{\textrm{[GOAL] [GOAL] ... [GOAL]}}_{d-k}}^{d}
\end{equation*}

$k$ 以下の目が出て,$i$ 番目の nest マスに落ちてしまう確率 $P(k\textrm{以下})$ は各々 $1/d$ です.一方,
$k$ よりも大きな目が出ていきなりクリアできる確率 $P(k\textrm{より大})$ は $1-k/d$ です.

これらから $\P$ はこう表せます.
\begin{eqnarray}
  \P&=&\left(\sum_{i=1}^kP(k\textrm{以下})\cdot x \cdot P_{k-i,d}(x)\right) + P(k\textrm{より大})\\
    &=&\left(\sum_{i=1}^k\frac{1}{d}\cdot x \cdot P_{k-i,d}(x)\right) + 1-\frac{k}{d}\\
    &=&\frac{x}{d} \left(\sum_{i=1}^kP_{k-i,d}(x)\right) + 1-\frac{k}{d}
\end{eqnarray}

総和を含む漸化式になってしまっているので差分を取ることで普通の漸化式にします.
\begin{eqnarray}
  \P-P_{k-1,d}&=&\frac{x}{d}P_{k-1,d}-\frac{1}{d}\\
  \P&=&\left(1+\frac{x}{d}\right)P_{k-1,d}-\frac{1}{d}\label{eqn:zen}
\end{eqnarray}

ここで$\P$,$P_{k-1,d}$ を $c$ に置き換えた式
\begin{eqnarray}
  c&=&\left(1+\frac{x}{d}\right)c-\frac{1}{d}\label{eqn:det}
\end{eqnarray}
を式 (\ref{eqn:zen}) から引くと
\begin{eqnarray}
  \P-c&=&\left(1+\frac{x}{d}\right)\left(P_{k-1,d}-c\right)
\end{eqnarray}
これは $\P-c$ に関する等比数列の式になっているので,
\begin{eqnarray}
  \P-c&=&\left(P_{1,d}-c\right)\left(1+\frac{x}{d}\right)^k\\
           &=&\left(1      -c\right)\left(1+\frac{x}{d}\right)^k
\end{eqnarray}
ところで方程式 (\ref{eqn:det}) を解くと解は $c=1/x$ なので
\begin{eqnarray}
  \P-\frac{1}{x}&=&\left(1-\frac{1}{x}\right)\left(1+\frac{x}{d}\right)^k\\
  \P            &=&\left(1-\frac{1}{x}\right)\left(1+\frac{x}{d}\right)^k+\frac{1}{x} \label{eqn:gen}
\end{eqnarray}
これで $\P$ の一般解\cite{rpakr}が求まりました.

\subsection{$k$ が大きいときにクリア確率が1未満になるnestマス数}
\begin{equation}
  \lim_{k \rightarrow +\infty}\max\left\{d \in \nat|\P<1\right\} \label{eqn:ans}
\end{equation}
を求めましょう.

$\Q = \P-x$ とおくと
\begin{eqnarray}
  \Q&=&\P - x\\
  \Q&=&\left(1-\frac{1}{x}\right) \left(1+\frac{x}{d}\right)^k + \frac{1}{x} - x\\
    &=&\left(1-\frac{1}{x}\right) \left(1+\frac{x}{d}\right)^k - \left(1-\frac{1}{x}\right)(1+x)\\
    &=&\left(1-\frac{1}{x}\right) \left(\left(1+\frac{x}{d}\right)^k -x-1\right)
\end{eqnarray}

$\R = (1+x/d)^k-x-1$ とおくと
\begin{eqnarray}
  \Q&=&\left(1-\frac{1}{x}\right) \R
\end{eqnarray}

$\R=0$ の解を調べます.$3<k<d$ とすると
\begin{eqnarray}
  R_{k,d}(0)&=&0\\
  R_{k,d}' (x) &=& \frac{k}{d}\left(1+\frac{x}{d}\right)^{k-1}-1\\
  R_{k,d}' (0) &=& \frac{k}{d}-1 < 0\\
  R_{k,d}''(x) &=& \frac{k(k-1)}{d}\left(1+\frac{x}{d}\right)^{k-2}>0
\end{eqnarray}
よって $R_{k,d}(1)>0$ ならその時に限り $\R=0$ は $0 \leq x < 1$ に解を持ちます(初期値からここに収束するかの議論は今回は省きます).
\begin{eqnarray}
  R_{k,d}'(1)&>&0\\
  \left(1+\frac{1}{d}\right)^k-2&>&0\\
  \left(1+\frac{1}{d}\right)^k&>&2\\
  k\ln\left(1+\frac{1}{d}\right)^k&>&\ln 2\\
  k&>&\frac{\ln 2}{\ln\left(1+\frac{1}{d}\right)} \cite{rpakr2}
\end{eqnarray}
$g(d)=\ln2/\ln(1+1/d)$ とおくと,
\begin{eqnarray}
  k&>&g(d)\\
  g(d) &=&\ln 2\cdot\ln^{-1}\left(1+\frac{1}{d}\right)\\
  g'(d)&=&\ln 2\cdot\left(-\ln^{-2}\left(1+\frac{1}{d}\right)\right)\left(\ln\left(1+\frac{1}{d}\right)\right)'\\
       &=&\ln 2\cdot\left(-\ln^{-2}\left(1+\frac{1}{d}\right)\right)\frac{1}{1+\frac{1}{d}}\left(\frac{1}{d}\right)'\\
       &=&\ln 2\cdot\left(-\ln^{-2}\left(1+\frac{1}{d}\right)\right)\frac{1}{1+\frac{1}{d}}\left(-\frac{1}{d^2}\right)\\
       &=&\ln 2\cdot\left( \ln^{-2}\left(1+\frac{1}{d}\right)\right)\left(d(d+1)\right)^{-1}\\
               &=&\frac{\ln 2}{        \ln\left(1+\frac{1}{d}\right)^d       \ln\left(1+\frac{1}{d}\right)^{d+1}}\\
  \liminf g'(d)&=&\frac{\ln 2}{\ln\liminf \left(1+\frac{1}{d}\right)^d\liminf\ln\left(1+\frac{1}{d}\right)^{d+1}}\\
               &=&\frac{\ln 2}{\ln e\cdot\ln e}\\
               &=&      \ln 2
\end{eqnarray}
傾きが $\ln 2$ に収束するんですね.

\begin{eqnarray}
  \liminf \left(g(d)-g'(d)d\right)&=&\liminf\left(\frac{\ln 2}{\left(1+\frac{1}{d}\right)}-d\ln 2\right)\\
                                  &=&\frac{\ln 2}{2}\\
  \liminf \left(g(d) - \left(d\ln 2 + \frac{\ln 2}{2}\right)\right) &=&0
\end{eqnarray}
よって,確率 1 でクリア出来るネストすごろくの最大の nest マスの数は,サイコロの目の数 $d$ を増やしていくと $d \ln2+\ln 2/2=0.6931\cdots+0.3465\cdots$ に近づきます.およそ $d$ の $69 \%$ 程度の値に収束するんですね.

\section*{Special thanks}
rpakr さん

\subsubsection{余談}
問 1\cite{nestsugoroku} を出した時に,ゆる言語学ラジオサポーターコミュニティの Miu さん \cite{miu} にも,多項式 $P_6(x)$ が下記のように書けると教えてもらいました.

\begin{eqnarray}
P_6(x)&=&\frac{1}{6}+\frac{1}{6}\sum_{k=1}^5(k+1)\frac{(6+x)^{k-1}}{6^k}\\
      &=&\frac{1}{x}\left((x-1)\left(1+\frac{x}{6}\right)^5+1\right)
\end{eqnarray}
式 (\ref{eqn:gen}) の $P_{5,6}(x)$ に一致しますね.

\begin{thebibliography}{99}
\bibitem{nestsugoroku} koteitan, "ネストすごろく", twitter, 2022.12.19 \url{https://twitter.com/koteitan/status/1604851675604209664}
\bibitem{nestsugoroku2} koteitan, "ネストすごろく2", twitter, 2022.12.24 \url{https://twitter.com/koteitan/status/1606485685325934593}
\bibitem{rpakr} rpakr, \href{https://discordapp.com/invite/3XK4tq6}{グーゴロジストの社交場 discord サーバー}, 森羅万象チャンネルネスト双六スレッド, \href{https://discord.com/channels/444127369033416704/1053714648727310336/1053948216791552100}{2022.12.18}
\bibitem{rpakr2} rpakr, \href{https://discordapp.com/invite/3XK4tq6}{グーゴロジストの社交場 discord サーバー}, 森羅万象チャンネルネスト双六スレッド, \href{https://discord.com/channels/444127369033416704/1053714648727310336/1053952001911832586}{2022.12.18}
\bibitem{miu} Miu, \href{https://yurugengo.com/}{ゆる言語ラジオ}\href{https://yurugengo.com/support}{サポーターコミュニティ discord サーバー}, その他科学・工学ラジオチャンネル ゆる数学ラジオスレッド, 2022.12.21
\end{thebibliography}
 
\end{document}
